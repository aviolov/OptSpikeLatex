\documentclass{article}
\usepackage{amsmath}
\usepackage{amscd}
\usepackage{amssymb}
\usepackage{amsfonts}
\usepackage{amsthm}
\usepackage{amsfonts}
\usepackage{amsthm} 
%\usepackage{circuitikz}
% \usepackage{pgf}
% \usepackage{tikz} 
% \usetikzlibrary{arrows,snakes,backgrounds}
% \usetikz
% \usepackage{verbatim} 
\usepackage{subfig} 
\usepackage{graphicx}
\newtheorem*{rep}{Reply}
% \usepackage[super]{nth}
% \usepackage{appendix}
% \usepackage{listings}
% \usepackage{color}

\usepackage{hyperref}
\usepackage{url}

\usepackage{cleveref}
\usepackage{aviolov_style}
\usepackage{local_style}
 
\begin{document}

\title{Spike-Controlling the Noisy ML Model} 

\author{Alexandre Iolov, 
Susanne Ditlevsen 
and
Andr\'e Longtin
}
\date{\today}

\maketitle 

\tableofcontents

\section{Stochastic Morris-Lecar Model}
\subsection{Diffusion model}
\def \Vt {{ U_s }} \def \Wt {{ W_s }} \def \Vz {{ U_z}} \def \Wz {{ W_z}} The
stochastic Morris-Lecar model including both current and channel noise is
defined as the solution to
\begin{equation}
\left\{
\begin{array}{ccl}
d\Vt &=& f(\Vt,\Wt)dt +\gamma d\tilde{B}_s,\\
d\Wt&=&b(\Vt,\Wt) dt  \, + \sigma(\Vt,\Wt)dB_s,
\end{array}
\right.
\tag{ML}
\label{eq:ML}
\end{equation}
where 
\begin{eqnarray*}
f(\Vt,\Wt)&=&\frac{1}{C}\Big(-g_{Ca}m_\infty(\Vt) (\Vt-V_{C_a})
\\ && - g_K\Wt(\Vt-V_K)
-g_L(\Vt-V_L)+I + A(s) \Big),
\\ b(\Vt,\Wt)&=&
\left(\alpha(\Vt)(1-\Wt) - \beta(\Vt)\Wt\right),\\ m_\infty(v)&=&\frac{1}{2}\left(1+\tanh\left(\frac{v-V_1}{V_2}\right)\right),\\
\alpha(v) &=& \frac{1}{2}\phi \cosh\left(\frac{v-V_3}{2V_4}\right)\left(1+\tanh\left(\frac{v-V_3}{V_4}\right)\right),\\
\beta(v) &=& \frac{1}{2}\phi \cosh\left(\frac{v-V_3}{2V_4}\right)\left(1-\tanh\left(\frac{v-V_3}{V_4}\right)\right),
\end{eqnarray*}

and the initial condition $(\Vz, \Wz)$ is random  with density $p(\Vz, \Wz)$.
 Processes $(\tilde{B}_s)_{s\geq s_0}$ and $(B_s)_{s\geq s_0}$ are independent
 Brownian motions.
The variable $\Vt$ represents the membrane potential of the neuron at time $s$,
and $\Wt$ represents the normalized conductance of the K$^+$ current. It varies
between 0 and 1, and can be interpreted as the probability that a K$^+$ ion
channel is open at time $t$. The equation for $f(\cdot)$ describing the dynamics
of $V_t$ contains four terms, corresponding to Ca$^{2+}$ current, K$^+$ current,
a general leak current, the input current $I$ AND THE EXTERNAL CONTROL CURRENT
$A(s)$!
  
 The functions $\alpha
  (\cdot )$ and $\beta (\cdot )$ model the rates of opening and closing of the
  K$^+$ ion channels. The function $m_{\infty}(\cdot)$ represents the
  equilibrium value of the normalized Ca$^{2+}$ conductance for a given value of
  the membrane potential. The parameters $V_1, V_2, V_3$ and $V_4$ are scaling
  parameters; $g_{Ca}, g_K$ and $g_L$ are conductances associated with
  Ca$^{2+}$, K$^{+}$ and leak currents; $V_{Ca}, V_K$ and $V_L$ are reversal
  potentials for Ca$^{2+}$, K$^+$ and leak currents; $C$ is the membrane
  capacitance; $\phi$ is a rate scaling parameter for the opening and closing of
  the K$^+$ ion channels; and $I$ is the input current.


\begin{center}
\begin{figure}
\subfloat[Type 1]{
\includegraphics[width=.5\textwidth]{Figs/MLBox/ML_type1_example_traj.pdf}
\label{fig:simu_ML_1} 
}
\subfloat[Type 2]{
\includegraphics[width=.5\textwidth]{Figs/MLBox/ML_type2_example_traj.pdf}
\label{fig:simu_ML_2}
}

\caption{Simulated trajectory of the stochastic 
  Morris-Lecar model: $(\Vt)$ as a function of time (top),
  $(\Wt)$ as a function of time (bottom). Parameters are given in Section
  \ref{sec:simulation}. Time is measured in ms, voltage in mV,
  the conductance is normalized between 0 and 1. }
 \label{fig:simu_ML} 
\end{figure}
\end{center}

Various noise sources are present in single neurons, and they act on many
different spatial and temporal scales \citep{Gerstner2005,Longtin2013}. A main
component arises from the synaptic bombardment from other neurons in the
network, and in the diffusion limit appears as an additive noise on the current
equation. Parameter $\gamma$ scales this current noise. Conductance fluctuations
caused by random opening and closing of ion channels leads to multiplicative
noise on the conductance equation. Function $\sigma(\Vt,\Wt)$ models this
channel or conductance noise. We consider the following function that ensures
that $\Wt$ stays bounded in the unit interval  if $\sigma \leq 1$
\citep{DitlevsenGreenwood2011}: $$\sigma(\Vt, \Wt ) = \sigma \sqrt{2
\frac{\alpha(\Vt) \beta(\Vt)}{\alpha(\Vt)+ \beta(\Vt)} \Wt (1-\Wt)}. $$
A strajectory of the model is simulated in Fig. \ref{fig:simu_ML}. The peaks of
$(\Vt)$ correspond to spikes of the neuron.

\subsection{Simulation study}\label{sec:simulation}
Parameter values of the Morris-Lecar model used in the simulations are the same
as those of \citep{RinzelErmentrout1989,Tateno2004} for a class I and for a
class II membrane.
 
The values are:\\ 
% \begin{verbatim}
\input{Figs/MLBox/MLParams_type1.txt}
% \end{verbatim}
AND \\
% \begin{verbatim}
\input{Figs/MLBox/MLParams_type2.txt}
% \end{verbatim}

% $V_K = -84$ mV, $V_L = -60$ mV, $V_{Ca} = 120$ mV, $C = 1\,
% \mu$F/cm$^2$, $g_L = 0.1\, \mu$S/cm$^2$, $g_{Ca} = 0.22 \,
% \mu$S/cm$^2$, $g_K = 0.4\, \mu$S/cm$^2$, $V_1 = -1.2$ mV, $V_2 = 18$ mV,
% $V_3 = 2$ mV, $V_4 = 30$ mV, $\phi = 0.04$ $\mbox{ms}^{-1}$.  
% Input is chosen to be $I = 90 \, \mu$A/cm$^2$.

(The values are from Susanne's code, in Andre's code the values are normalized
st $C= 1$ and there is no noise, whereas in Susanne's code the parameter values
for $\s, \g$ are given and so I don't have to rederive them)

Initial conditions for the Morris-Lecar model are $V_{t_0} = -26$ mV, $W_{t_0} =
0.2$. 

Trajectories are simulated with time step $\Delta
t=0.01$ ms.  An example of a simulated trajectory is
given in Figure \ref{fig:simu_ML}.

IMPORTANT: From now on we will use data generated from a TYPE I (One) Model, as
it is deemed more suitable to be well-fitted by an LIF model.


\section{Morris Lecar $<->$ LIF conversion}
We consider how to go about converting $V,W,s$ data from the Morris-Lecar model
into $Y,s$ data from the (dimensional) LIF model,
 
\begin{equation}
\begin{gathered}
dY(s) = \frac 1C \left( m + A(s) - Y  \right) \intd{s} + \s \intd{B_s},
\\
Y(\ss) = \yth \implies
\begin{cases}
Y(\ss^+) = \yreset & 
\end{cases}
\end{gathered}
\tag{OU:dim}
\label{eq:LIF_dimenisonal}
\end{equation}
Which in turn can be transformed into the non-dimensional LIF model, which is
used to determine the control algorithm:
\begin{equation}
\begin{gathered}
dX(t) = \left(\m + \a(t) - \frac{X(t)}{\tc} \right) \intd{t} + \b \intd{B_t},
\\
X(\ts) = 1 \implies
\begin{cases}
X(\ts^+) = 0 & 
\end{cases}
\end{gathered}
\tag{OU:non-dim}
\label{eq:LIF_non_dimenisonal}
\end{equation}
We use the same symbol $C$ in both \cref{eq:ML} and \cref{eq:LIF_dimenisonal}
because the two models never really 'talk' to each other through their parameters. The
two models only see each other through their data, 
In particular given data from \cref{eq:ML} we estimate the
parameters $C, m,\s, \yth, \yreset$ in \cref{eq:LIF_dimenisonal}. But we never
estimate (know) the parameter $C$ in the Morris-Lecar model, in fact as far as
we are concerned, the M-L data comes from a Black Box. 

We will need to know how to perform the following tasks:
\begin{enumerate}
  \item Given $V,s$ how to estimate $m, Cc, \s$ (In general estimating
  Ornstein-Uhlenbeck parameters is fairly straightforward, but only given OU
  data)
  
  \item determine the physical threshold and reset $\yth, \yreset$
  
  \item Given an optimal control $\a(x,t)$ determined in the $X,t$ domain,
  convert to $A(Y, s)$ in the $Y,s$ domain. (And then apply $A$ to $V$ in the
  ML model).
\end{enumerate}

\subsection{Converting the OU model from dimensional to non-dimensional form
and back}
Let's start with the translation between $Y,s$ and $X,t$. First we
normalize $X$ to reset to 0 and spike at $1$. With that we have that $X_t =
\frac{Y_s - \yreset}{\yth - \yreset} $. We also non-dimensionalize time, using $\meanI$,
which is the mean observed ISI. (Alternatively, we could normalize by the mean
ISI of the target train. We'll use the same symbol in any case) $$ t = \frac
s\meanI$$


Let us write, $G$ for gap.
$$G = \yth - \yreset$$
b/c the expression $\yth - \yreset$  occurs quite often. 

Now we can do the conversions:
\begin{eqnarray*}
\Delta Y =&
	\frac 1 C (m + A - Y) \Delta s +
 	\s \sqrt{\Delta s} \cdot \xi \\
\frac {\Delta Y}{G} =&
	\frac {\meanI}{GC} (m - \yreset + A - (Y-\yreset))\frac{\Delta s}{\meanI} +
	\frac{\s}{G} \cdot \sqrt{\meanI} \sqrt{\frac{\Delta s}{\meanI}} \cdot \xi
\\
\Delta X  = &
	(\frac{\meanI (m - \yreset)}{GC} + \frac {\meanI\cdot A}{GC} -
	\frac{\meanI}{C} X) \Delta t +  
	\frac{\s}{G} \cdot \sqrt{\meanI} \sqrt{\Delta t} \cdot \xi
\end{eqnarray*}
From this we can read off, $\m, \tc, \b, \a$. We have:
\begin{subequations}
\begin{eqnarray}
X_t = \frac{Y_s - \yreset}{\yth - \yreset} 
& \iff  &
 Y_s =  \ldots
\\ 
t = \frac s \meanI &
\iff & s = \meanI t
\\
\m = \frac{\meanI (m - \yreset)}{C\cdot G}
& \iff &
m = \ldots   
\\
\a() = \frac{\meanI \cdot A()}{GC}
&\iff&
A() = \frac{GC \cdot \alpha() }{\meanI}
\\
\tc = \frac{ C }{\meanI }
&\iff&
C = \ldots
\\
\b = \frac{\s \sqrt{\meanI} }{\yth - \yreset}
&\iff&
\s =\ldots
\end{eqnarray}
\end{subequations}

We've used \ldots to indicate conversions that will never be made, so they do
not matter. (We will never go $\mu \ra m$).

To indicate that we've done everything correct we show two trajectories $Y_s,
X_t$ in \cref{fig:OU_dim2nondim_conversions}.
%\usepackage{graphics} is needed for \includegraphics
\begin{figure}[htp]
\begin{center}
  \includegraphics[width=\textwidth]{Figs/MLBox/OUDim2NonDimConversion.pdf}
  \caption[labelInTOC]{Conversions between $Y_s\ra X_t$. 
  Top plot is a chunk of $Y$ data. Below is a chunk of $X$ data using the conversion params.
  Also below we plot the original $Y$ chunk from above converted to $X$.
  Obviously we've used the same stream of random numbers. The parameters work
  out to be:
  meanISI = 189.11,\\
$m = -16.91 C = 132.72 \sigma = 1.00$ \\ 
$\mu = 1.39 \tc = 0.70 \beta = 0.40 $.\\ 
The physical parameters, $m,C,\sigma $ are obtained from
\cref{sec:estimaing_LIF_from_ML}}
\label{fig:OU_dim2nondim_conversions}
\end{center}
\end{figure} 


\subsection{Estimating the LIF params from ML data}
\label{sec:estimaing_LIF_from_ML}
This estimation and in general the relation between $Y$ and $V$ is the most
challenging aspect of the whole problem!

It is difficult to unambiguously determine what if anything is the threshold,
the point-of-no-return for $V$ in \cref{fig:simu_ML}. But it looks to be about
-14.0. In addition we stipulate a lower threshold, at -22, which is the point
after which we no longer consider the data as suitable to be represented as an
OU process. Thus rises the question of what to do between (-22 and -14). We will
say that this is a grey zone where the diffusion has almost spiked: $X = 1-
\epsilon$, say $X = .999$

Let's plow ahead! \Cref{fig:ML_estimates_data} shows the original $V,s$ data,
plus an identification of the spikes, using a completely arbitrarily chosen
threshold, $\yth = -14$. A spike is recorded when $V$ (equivalently $Y$) crosses
$\yth$. 

Once a spike is recorded, we wait an absolute refractory time of $$t_{ref} =
40 ms$$ at which point we restart. As such the time between $\ts$ and
$\ts+t_{ref}$ is compactified to a point as far as the LIF is concerned. I've
played around with different values of $t_{ref}$ and $40 ms$ seems to give the
best results. In particular notice on the bottom plot in
\cref{fig:ML_estimates_data} how all trajectories converge more or less to the
same point. Visually, this uniform reset effect is achieved for any $t_{ref} \in
[37, 43]ms$.

The data between the reset time and the OU bound crossing time is used for
parameter estimation. (It is plotted in the middle plot in \cref{fig:ML_estimates_data})

\begin{figure}[htp]
\begin{center}
  \includegraphics[width=\textwidth]{Figs/MLBox/ML_estimates_annotated.pdf} 
  \caption[labelInTOC]{Annotating (spike-segmenting) the Raw ML data. 
  This is the same data as in \cref{fig:simu_ML_1} (redrawn in top plot).
  The middle plot shows the data segments that are used to estimate the OU
  parameters from ML data.
  The bottom plot shows the stereotypical spike shape and justifies why 40 ms
  is a reasonable refractory time}
  \label{fig:ML_estimates_data} 
\end{center}
\end{figure} 

Now we want to estimate $m, C, \s$ from a chunk of Morris-Lecar generated data.

Let's recall the MaxLikelihood formulas for the parameters of the non-threshold,
$A\equiv 0$ OU process in \cref{eq:LIF_dimenisonal} given discrete
observations, $\{Y_n\}_{n=0}^{N}$:
\begin{subequations}
\begin{eqnarray}
\hat{ m} &=&  \frac{ \sum_{n=1}^{N} \left( Y_n - e^{-\frac {\Delta} {\hat C}}
Y_{n-1} \right)} {N ( 1-e^{-\frac {\Delta} {\hat C}})}
\\
 0 &=& e^{-\frac {\Delta}{\hat{C}} } -
  \frac { \sum_{n=1}^{N} ( Y_n - \hat m)( Y_{n-1} - \hat m) }
  	   { \sum_{n=1}^{N} \left( Y_{n-1} - \hat m \right)^2 } 
\\
\hat\s^2 &=& 
2\frac{ \sum_{n=2}^{N}  \left( Y_n - \hat m - (Y_{n-1} -
\hat m) e^{-\frac {\Delta} {\hat C}} \right)^2 } 
	  {N \cdot (1-e^{-2\frac {\Delta} {\hat C}}) \hat C}
\end{eqnarray}
\end{subequations}
This is for one uninterupted sample. However, we have $K$ distinct paths
each consisting of $N_k+1$ data points ($N_k$ transitions). Thus we must
generalize the formulas:
\begin{eqnarray*}
\hat{ m} &=& 
\frac{ \sum_{k=1}^K \sum_{n=1}^{N_k} \left( Y^{(k)}_n - e^{-\frac{\Delta_n} {\hat C}} Y^{(k)}_{n-1} \right)} 
	 { N( 1-e^{-\frac {\Delta_n} {\hat C})}}
\\
0 &=& e^{-K\frac {\Delta}{\hat{C}} } - 
\prod_{k=1}^K\left[ \frac { \sum_{n=1}^{N_k} ( Y^{(k)}_n - \hat m)(Y^{(k)}_{n-1} - \hat m) }
						  { \sum_{n=1}^{N_k} \left( Y^{(k)}_{n-1} - \hat m \right)^2 } 
  	    \right]
\\
\hat\s^2 &=&  
2\frac{ \sum_{k=1}^K \sum_{n=2}^{N}  \left( Y^{(k)}_n - \hat m - (Y^{(k)}_{n-1} -
\hat m) e^{-\frac {\Delta} {\hat C}} \right)^2 } 
	  { N (1-e^{-2\frac {\Delta} {\hat C}}) \hat C}
\end{eqnarray*}
Where $N= \sum_k N_k$
  
Now we are ready to estimate $m, C, \s$ given data from the ML trajectories.
Given a stream of ML-Type 1 data points, we get the estimates in
\cref{tab:OU_from_ML1_ests}.
\begin{table}
\begin{tabular}{l|cc}
\input{Figs/MLBox/OU_from_ML1_estimates.txt}
\end{tabular}
\caption{Estimates for the three OU parameters $m, C, \s$ given data from an ML
type 1 model (middle-column) and then from an OU model ran with the estimated
parameters of the ML 1 model (right column)}
\label{tab:OU_from_ML1_ests}
\end{table}
Then, if we run the OU model with the inferred parameters from
\cref{tab:OU_from_ML1_ests} (the ML1 column), we get the OU trajectories in
\cref{fig:OU_vs_ML1}. Finally, recalling the conversions of $Y$ to $X$ to
non-dimensionalize the OU process gives us the values for $\m,\tc, \s$ in
\cref{tab:OU_nondim_vs_OU_dim}. This is a very interesting parameter regime. 
It is sub-threshold, $\bar X \approx .66$, but the noise strength is halfway
between what we called Low-Noise $\b=.3$ and High-Noise $\b=1.5$. We know our
control is very good 

In \cref{tab:OU_nondim_vs_OU_dim}, we can also consider the bounds on our
control. We've assumed that $$A(s) \in [\pm 20]$$. We see that this gives
us quite wide bounds on $\a(\cdot)$. Recall that so far we have used $\a \in
[-2,2]$ and we've only needed more 'oomph' in the case of high-noise, i.e.\
$\b=1.5$, whereas here we have the milder $\b=0.7$. Also the time-constant is
pretty small, meaning, we have a pretty stiff system, i.e.\ one which quickly
mean-reverts to the base-line $0$.


\begin{figure}[htp]
\begin{center}
  \includegraphics[width=\textwidth]{Figs/MLBox/OU_from_ML_type1_Tf2000.pdf}
  \caption[labelInTOC]{The original Type 1 ML sub-threshold dynamics vs.\ OU
  dynamics. Notice that the OU dynamics are generated using the exact same
  $Y_0$ and the SAME stream of random numbers as those that generate the ML1
  dynamics. It is clear that the match (OU of ML1) is satisfactory.} 
  \label{fig:OU_vs_ML1}
\end{center} 
\end{figure}

\begin{table}
\begin{align*}
\input{Figs/MLBox/OUDims2Nondims.txt}
\end{align*} 
\caption{Estimates for the three OU parameters $m, C, \s$ and their 
non-dimensional counterparts, $\mu , \tc ,\beta$} 
\label{tab:OU_nondim_vs_OU_dim}
\end{table}

\clearpage

\section{Controlling the Morris-Lecar model}
\label{sec:controlling_ML}
We now have all the pieces to control the Morris-Lecar model. Here is a summary:
\begin{enumerate}
  \item Before doing anything estimate the LIF params, $m, C, \s$ (we've
  already done this) and their conversions to non-dimensional form $\mu , \tc
  ,\beta$
  \item Take a series of target spike times, we will use the spikes generated by
  the model in 
  \item Given $V_s$, treat it like $Y_s$ and convert, $Y_s$ into $X_t$. Convert
  the remaining time-to-spike from $s \ra t$. 
  \item determine the optimal $\a(t, X_t)$ from closed or open-loop. 
  \item convert $\a()$ into $A(s)$
  \item Apply $A(s)$ to the $dV_s$ system.
  \item Rinse and repeat  
\end{enumerate} 

Ok, let's skip to the chase - the results are in
\cref{fig:ML_controled_simulation_cl} and \cref{fig:ML_controled_simulation_ol}.
It is clear that the control strategy (for closed-loop) is highly accurate! This
might have much to do with the fact that we have assumed such wide $A$ bounds.
$$A(s) \in [\pm 20]$$ $$\implies \a(t) \in [\pm 4.7]$$


Should we reduce the power bounds\ldots? In general how should we come up with
the values for $A_{min}, A_{max}$?!

For reference
\cref{fig:ML_controled_simulation_cl_A10,fig:ML_controled_simulation_ol_A10}
give the results for $$A_{min}, A_{max} = [-10,10]$$.

%%%%%%%%%%%% CL %%%%%%%%%%%%%%%%%%%%%%%%%%%%%%%
\begin{figure}[h]
\begin{center}
\subfloat[example controlled trajectory]
{
\label{fig:ML_example_controled_traj}
\includegraphics[width=\textwidth]
{Figs/MLTrainController/MLSim_example_cl.pdf} 
}\\
\subfloat[raster plot for controlled simulations]
{
\label{fig:ML_raster_plot}
\includegraphics[width=\textwidth] 
{Figs/MLTrainController/MLSim_raster_cl.pdf}
}
\caption[labelInTOC]{Example of a controlled ML trajectory and the raster plot
of realized spikes vs.\ controlled spikes, $N=5$ 	(Closed-Loop!)}
\label{fig:ML_controled_simulation_cl}
\end{center} 
\end{figure}
%%%%%%%%%%%% OL %%%%%%%%%%%%%%%%%%%%%%%%%%%%%%%
\begin{figure}[h] 
\begin{center}
\subfloat[example controlled trajectory]
{
\label{fig:ML_example_controled_traj}
\includegraphics[width=\textwidth]
{Figs/MLTrainController/MLSim_example_ol.pdf} 
}\\
\subfloat[raster plot for controlled simulations]
{
\label{fig:ML_raster_plot}
\includegraphics[width=\textwidth]
{Figs/MLTrainController/MLSim_raster_ol.pdf} 
}
\caption[labelInTOC]{Example of a controlled ML trajectory and the raster plot
of realized spikes vs.\ controlled spikes, $N=4$ 	(Open-Loop!)}
\label{fig:ML_controled_simulation_ol}
\end{center} 
\end{figure}
%%%%%%%%%%%%%%%%%%%%%%%%%%%%%%%%%%%%%%%%%%%%%%%%%%%%%
%%% REDUCED POWER BOUNDS: %%%%%%%%%%%%%%%%%%%%%%%%%%%
%%%%%%%%%%%% CL %%%%%%%%%%%%%%%%%%%%%%%%%%%%%%%%%%%%%
\begin{figure}[h]
\begin{center}
\subfloat[example controlled trajectory]
{
\label{fig:ML_example_controled_traj}
\includegraphics[width=\textwidth]
{Figs/MLTrainController/MLSim_example_cl_Amax10.pdf} 
}\\
\subfloat[raster plot for controlled simulations]
{
\label{fig:ML_raster_plot}
\includegraphics[width=\textwidth] 
{Figs/MLTrainController/MLSim_raster_cl_Amax10.pdf}
}
\caption[labelInTOC]{Same as \cref{fig:ML_controled_simulation_cl}
(closed-loop) but with reduced power bounds $A \in [-10,10]$}
\label{fig:ML_controled_simulation_cl_A10} 
\end{center}
\end{figure}
%%%%%%%%%%%% OL %%%%%%%%%%%%%%%%%%%%%%%%%%%%%%%
\begin{figure}[h] 
\begin{center}
\subfloat[example controlled trajectory]
{
\label{fig:ML_example_controled_traj}
\includegraphics[width=\textwidth]
{Figs/MLTrainController/MLSim_example_ol_Amax10.pdf} 
}\\
\subfloat[raster plot for controlled simulations]
{
\label{fig:ML_raster_plot}
\includegraphics[width=\textwidth]
{Figs/MLTrainController/MLSim_raster_ol_Amax10.pdf} 
}
\caption[labelInTOC]{Same as \cref{fig:ML_controled_simulation_ol}
(open-loop) but with reduced power bounds $A \in [-10,10]$}
\label{fig:ML_controled_simulation_ol_A10}
\end{center}
\end{figure}

\clearpage

\section{Morris Lecar $<->$ LIF conversion - Spikes-Only}
It remains to do one more thing - estimate the LIF parameters from Spike-Data
only. 

However, we know that estimating $\tc$ from spikes is problemeatic. To keep
things simple at first. We assume we know $\tc$ (we'll use the estimated value
from the \cref{tab:OU_from_ML1_ests}) and only estimate $\m, \b$, the mean
bias current and the noise intensity.

We use the Ditlevsen-Lansky scheme that utilizes the Fortet eqn to estimate
$\m, \b$ - \cite{Ditlevsen2007}.

In short, first we scale time by $C$ and we introduce a unit-tau-char variable
$Z$
$$
dZ = (a- Z) dt + bdW_t$$
with the relation between $a,b$ being
$$
a = \frac{m - \yreset}{\yth - \yreset} 
	\quad \iff \quad 
		 m = a \cdot (\yth - \yreset) + \yreset
$$
$$
b  = \frac{\sigma \sqrt{C}}{\yth - \yreset} 
	\quad \iff \quad  
		\s = \frac{b \cdot (\yth - \yreset)}{\sqrt{C}} 
$$

We estimate $a,b$ by minimizing the Fortet error and off we go. The estimates
are given in \cref{tab:OU_from_ML1spikes_ests} ($N=92$ spikes (20 secs.)). We
see that the estimates are quite accurate meaning that the spikes only estimates are no more
than $10$\% off from the estimates using detailed trajectories (in
\cref{tab:OU_from_ML1_ests}.

It remains to be seen whether this error has any discernible effect on the
control performance (It is plausible that the performance is imporved)

An aside/caveat: the Fortet estimator seems to be quite sensitive to
initial conditions, meaning that if you give it a too small guess for $\s$ is estimates
the noise intensity as $0$ (Possible bug of mine, I don't know).
\begin{table}
\begin{tabular}{l|c}
\input{Figs/MLBox/OU_from_ML1_estimates_spikes_only.txt}
\end{tabular}
\caption{SPIKES-ONLY: Estimates for the three OU parameters $m, C, \s$ given
data from an ML type 1 model (middle-column) and then from an OU model ran with the estimated
parameters of the ML 1 model (right column)}
\label{tab:OU_from_ML1spikes_ests}
\end{table}

\begin{table}
\begin{align*}
\mu &= 2.99 &\tc &= 0.22 &\beta &= 0.70
\\
\mu &=3.64  &\tc &= 0.22  &\beta &= 0.57
\end{align*} 
\caption{Estimates for the three non-dim OU parameters $\mu , \tc ,\beta$
Top: Voltage tracings, Bottom (Spikes-Only). recall that $\tc$ is artificially
set in the spikes only, which is why the two are the same!}
\label{tab:OU_nondim_voltage_vs_spikesonly}
\end{table}

\subsection{ML Control based on Spike-only estimates}
The final thing to do is control the ML model using the open-loop controller
based on the spikes only measurements. (It seems unnecessary to show the
performance of the closed-loop controller on spikes-only estimates - If you
can close-loop control, then you can voltage-estimate\ldots)

The performance is shown in
\cref{fig:ML_controled_simulation_ol_A10_spikesonly}. 

Now Susanne suggested that the arbitrary, un-identifiable value of $C$ be fixed
to something external (from 'above'). Ok, let's do this, we set $C=20$ and show
the open-loop results in
\cref{fig:ML_controled_simulation_ol_A10_spikesonly_C20}

%%%%%%%%%%%% OL %%%%%%%%%%%%%%%%%%%%%%%%%%%%%%%
\begin{figure}[h] 
\begin{center}
\subfloat[example controlled trajectory] 
{
\label{fig:ML_example_controled_traj}
\includegraphics[width=\textwidth]
{Figs/MLTrainController/MLSim_example_ol_Amax10_spikesonly.pdf} 
}\\
\subfloat[raster plot for controlled simulations]
{
\label{fig:ML_raster_plot}
\includegraphics[width=\textwidth]
{Figs/MLTrainController/MLSim_raster_ol_Amax10_spikesonly.pdf} 
}
\caption[labelInTOC]{Same as \cref{fig:ML_controled_simulation_ol_A10}, but now
the parameters are estimated using spikes-only ($N=92 $spikes for estimation
$T=20$ secs.)}
\label{fig:ML_controled_simulation_ol_A10_spikesonly}
\end{center}
\end{figure}
\begin{figure}[h] 
\begin{center}
\subfloat[example controlled trajectory] 
{ 
\label{fig:ML_example_controled_traj}
\includegraphics[width=\textwidth]
{Figs/MLTrainController/MLSim_example_ol_Amax10_spikesonly_C20.pdf} 
}\\
\subfloat[raster plot for controlled simulations]
{
\label{fig:ML_raster_plot}
\includegraphics[width=\textwidth]
{Figs/MLTrainController/MLSim_raster_ol_Amax10_spikesonly_C20.pdf} 
}
\caption[labelInTOC]{Same as
\cref{fig:ML_controled_simulation_ol_A10_spikesonly}, but now the $C$ parameter
is fixed at $20.0$ (millisecs/ohms?) instead of using the voltage-based
estimation (when $C=39.46$ see \cref{tab:OU_from_ML1_ests}))}
\label{fig:ML_controled_simulation_ol_A10_spikesonly_C20}
\end{center}
\end{figure}

\bibliographystyle{plain}
\bibliography{library}
\end{document}