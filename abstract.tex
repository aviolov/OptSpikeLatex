\documentclass[12pt]{iopart}
%\documentclass{article}

\usepackage{amsmath}
\usepackage{amscd}
\usepackage{amssymb} 
\usepackage{amsfonts}     
\usepackage{amsthm}
\usepackage{amsfonts} 
\usepackage{amsthm} 

\usepackage{tikz}
\usetikzlibrary{arrows,snakes,backgrounds}
% \usepackage{circuitikz}
% \usepackage{pgf}
% \usetikz
\usepackage{subfig}

\usepackage{algpseudocode}
\usepackage{algorithm}

\usepackage[super]{nth}
\usepackage{appendix}
% \usepackage{listings}
\usepackage{color}

\usepackage{hyperref}
%\usepackage{url}

\usepackage{cleveref}
\usepackage{cancel}
\usepackage{slashbox}

\usepackage{aviolov_style}
\usepackage{local_style}

\usepackage{iopams}

% \newtheorem{thm}{Theorem}[section]
% \newtheorem{lemma}{Theorem}[thm]
% \theoremstyle{definition}
% \newtheorem{ex}{Example}[thm]
% \newtheorem{defn}{Definition}[thm]

% \def\includegraphic{}
% \def\includegraphics{}
 
\begin{document} 


\title{Stochastic Optimal Control of Single Neuron Spike Trains}
\author{Alexandre Iolov$^{1,2}$, Susanne Ditlevsen$^2$, Andr\'e Longtin$^{1,3}$ 
		}	
\address{%
     (1) Department of Mathematics and Statistics, University of Ottawa, Ottawa,
    Canada
\\
    (2) Department of Mathematical Sciences, University of Copenhagen,
Copenhagen, Denmark
\\
    (3) Department of Physics, University of Ottawa,
    Ottawa, Canada
}
\ead{aiolo040@uottawa.ca}			
% \address{Department of Mathematics and Statistics, University of Ottawa, Ottawa, Canada 
% \\ Department of Mathematical Sciences, University of Copenhagen, Copenhagen,
% Denmark }
% \ead{$<$\href{mailto:aiolo040@uottawa.ca}
% 		{aiolo040 at uottawa dot ca}$>$}

% \date{\today}

\begin{abstract}
We have devised an algorithm for eliciting exact spike times from a noisy
neuron. The strategy is especially designed to deal with a significant
stochastic component in the impinging current and is applicable in both
Supra- and Sub-threshold regimes. Depending on the measurements available to
the controller, this involves solving a Hamilton-Jacobi-Bellman PDE or a pair of
Kolmogorov PDEs (Forward and Backward). We provide a numerical implementation of
the control strategy and test its performance in simulations for a variety of
possible parameter regimes. 

\end{abstract}

\maketitle

\section{Novelty and Significance}
External control of exact spike times in single neurons is important for
understanding neuronal activity and has potential in brain-machine interfaces,
for improving neural prostheses, and in medical applications to neuronal
disorders, such as Parkinson's disease or epilepsy. The goal of this paper is
the design of optimal controlled electrical stimulation patterns to a cell to
achieve a target spike train under physiological constraints not to damage
tissue. We pose a stochastic optimal control problem to precisely specify the
spike times in a noisy leaky integrate-and-fire model of a neuron. In
particular, we allow for noise of arbitrary intensity. The optimal control
problem is solved using Dynamic Programming if the controller has access to the
voltage (closed-loop control) and using a Maximum Principle for the transition
density if the controller only has access to the spike times (open-loop
control). The techniques are applicable in both the supra-threshold and
sub-threshold regimes. Given this scope, our simulations show that our
algorithms produce the desired results. They further reveal that the numerical
demands are non-trivial, such that the algorithms may require efficiency
refinements to achieve real-time control, in particular the open-loop context is
more numerically demanding than the closed-loop one. Furthermore, we show how
wrong the control strategy can perform if we assume that the noise is
negligible.  More precise targeting and control of neural activity to achieve
optimal performance in a noisy environment, in which most neurons are embedded,
have great potential in health care. The main contribution is the online
feedback control of a noisy system through modulation of the input, taking into
account physiological constraints on the control, as well as incurred costs to
diminish adverse effects.

\end{document}