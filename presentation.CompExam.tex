
%% LaTeX Beamer presentation template (requires beamer package)
%% see http://bitbucket.org/rivanvx/beamer/wiki/Home
%% idea contributed by H. Turgut Uyar
%% template based on a template by Till Tantau
%% this template is still evolving - it might differ in future releases!

% \documentclass[draft]{beamer}
% \documentclass[handout]{beamer}
\documentclass{beamer}
% \usepackage{pgfpages} %for two screen presentations
% \pgfpagesuselayout{2 on 1}[a4paper,border shrink=5mm] %for handout!
% \setbeameroption{show notes} 
% \setbeameroption{show notes on second screen}
% \setbeameroption{show next slide on second screen}
% \setbeameroption{second mode text on second screen=left}
\setbeamertemplate{navigation symbols}{}

\mode<presentation>
{
\usetheme{Warsaw} 
% \usetheme{Frankfurt}
% \usetheme{Berkeley}
\usecolortheme{seahorse}
% \usecolortheme{rose}
\usefonttheme[onlylarge]{structuresmallcapsserif}
\usefonttheme[onlysmall]{structurebold}

\setbeamercovered{transparent}
\setbeamercolor{title}{fg=blue!90!black,bg=red!70!white}
}

\usepackage[english]{babel}
\usepackage[latin1]{inputenc}

% font definitions, try \usepackage{ae} instead of the following
% three lines if you don't like this look
\usepackage{mathptmx}
\usepackage[scaled=.90]{helvet}
\usepackage{courier}

\usepackage[T1]{fontenc}

\usepackage{amsmath}
\usepackage{amscd}
\usepackage{amssymb}
\usepackage{amsfonts}
\usepackage{amsthm}
\usepackage{amsfonts}
\usepackage{amsthm}

\usepackage{tikz}
\usepackage{subcaption}
\usepackage{subfigure}

\usepackage{aviolov_style}
\usepackage{local_style}


\title[Spike-Control for Neurons]{Optimal Control of First-Passage Time
in an Ornstein-Uhlenbeck Model
\\
---
\\
PhD Comprehensive Exam }
% - Use the \inst{?} command only if the authors have different
%   affiliation.
\author{Alexandre Iolov}
\institute{University of Ottawa}
\date{January 10, 2013}

% % This is only inserted into the PDF information catalog. Can be left
% % out.
% \subject{Talks}

% If you have a file called "university-logo-filename.xxx", where xxx
% is a graphic format that can be processed by latex or pdflatex,
% resp., then you can add a logo as follows:
% \pgfdeclareimage[height=0.5cm]{university-logo}{university-logo-filename}
% \logo{\pgfuseimage{university-logo}}
% Delete this, if you do not want the table of contents to pop up at
% the beginning of each subsection:
% \AtBeginSubsection[]
% {
% \begin{frame}<beamer>
% \frametitle{Outline}
% \tableofcontents[currentsection,currentsubsection]
% \end{frame}
% }

% If you wish to uncover everything in a step-wise fashion, uncomment
% the following command:
%\beamerdefaultoverlayspecification{<+->}

\begin{document}

\begin{frame}
\titlepage
\end{frame}

% \section*{Outline}
\begin{frame}
\tableofcontents[pausesections]
\end{frame}

\section{Problem Background}
\subsection[LIFs]{Leaky Integrate-and-Fire Basics}
\begin{frame}
\frametitle{The Underlying Model}
% \framesubtitle{Subtitles are optional}
A voltage, $X_t$ autonomously decays to zero, is driven and spikes when it hits
threshold:
\begin{columns}[T]
\begin{column}{5cm}
\begin{equation*}
\begin{gathered}
dX_s = (\a(t) - \frac{X_s}{\tc} ) \intd{s} + \b \intd{W_s},
\\
X(0) = .0,
\\
X(\ts) = \xth \implies
\begin{cases}
\ts \sim \text{spike time} 
\\
X(\ts^+) = .0 &
\end{cases}
\end{gathered}
\end{equation*}
\end{column}
\begin{column}{5cm} 
\onslide<2->{
  \includegraphics[width=.99\textwidth]{Figs/sinusoidal_train_N=10_example.png}
}
\end{column}
\end{columns}

\pause
\begin{center}
$\a(.) \sim$ the control

$\a(t)\in [\amin,\amax]$
\end{center}

%\usepackage{graphics} is needed for \includegraphics
% \begin{figure}[htp]
% \begin{center}
%   \caption[labelInTOC]{Example Voltage Trajectory + Spikes}
%   \label{fig:voltage_spikes_example}
% \end{center}
% \end{figure}

% \note[item]{When $X$ hits $\vt$, we have a spike}
% \note[item]{$\a$ is our control and is completely determined by us}
\end{frame}

\subsection{Problem Objective}
\begin{frame}
\frametitle{Our Goal}
\only<1-2>{
Impose a pre-specified spike sequence,  $\{t_n\}_1^N$, on the neuron.
}
\onslide<2>{
\vskip 5pt

ASSUMPTION: We know when actual spikes, $\{t_{sp,n}\}$ occur

$\implies$ we can work sequentially:

$$
\T = t_n - t_{sp, n-1}
$$
}
\end{frame}

\begin{frame}
\frametitle{Our Goal}

Given a target $\T$ find $\a(t)$ minimizing $\Exp[(\T - \ts)^2]$
\begin{figure}
\begin{center}
\includegraphics[width=0.6\textwidth]
{Figs/ControlSimulator/DesiredVsRealized_Tsp}
\caption{ Controlled Trajectory Example
- observed vs.\ desired spike time 
}
\end{center}
\end{figure}

\end{frame}


\begin{frame}
% \frametitle{Formal Problem Statement}

\begin{block}{Formal Problem Statement}
Given, $\tc, \b$, $\T$ and $X_0 = .0$, find

\begin{equation}
\a(t) = \argmin_{\a(t)} \Big\{ \, J[\a] = 
\Exp\left[
(\ts - \T \big)^2 
+  
\e \int_0^\ts  \a^2(t) \intd{t}  \right]
\, \Big\}
\end{equation}
\end{block}
\end{frame}

\begin{frame}
\frametitle{Control Horizon}
Since we know if and when a spike occurs
$$
t>\T \implies \a(t) = \amax
$$ 
So our control problem takes place over the interval
$$  t \in [0 , \T] $$ 
\end{frame}

 
\section{Closed-loop stochastic control - dynamic programing}
\begin{frame}
\frametitle{Dynamic Programing Solution}
If we had state observations / feedback
\vskip2pt
$\implies$ Dynamic Programming /
Hamilton-Jacobi-Bellman (HJB) Equations
\vskip 3pt
\onslide<2>{
\begin{equation*}
\begin{gathered}
\di_t v(x,t) + \tfrac{\b^2}{2} \di_x^2 v + 
\e \a^2 + (\a -\tfrac{x}{\tc}) \di_x v 
= 0
\\
\a (x,t) = \min \left(\amax,
		 \max\left(\amin, -\frac{\di_x v(x,t)}{2\e}\right)
\right)
\\
\begin{cases}
v(\xth,t) = (s-\T)^2  \quad & \textrm{upper BC}
\\
\di_x v(\xmin, t)  = .0  \quad &\textrm{lower BC}
\\
v(x,\T)  = \Exp[ (\t^2 : X_{\T+\t} = \xth) \,|\,X_\T = x, \a(t) = \amax]  \quad&
\textrm{TC}
\end{cases}
\end{gathered}
\label{eq:OC_LS_HJB_full}
\end{equation*}
\note[item]{explain the Terminal Condition - if we hit time $\T$ and nothing
has happened we smash on the gas}
}
\end{frame}


% \begin{frame}
% Simplest solution $\sim$ ignore the noise!
% 
% $\implies$ Deterministic Optimal Control (in 1-D)
% 
% 
% \begin{figure}
% \begin{center}
%   \includegraphics[width=.6\textwidth]{Figs/ControlSimulator/deterministic_example.png}
%   \caption{An example trajectory for the deterministic dynamics. }
% \end{center}
% \end{figure}
% 
% \end{frame}
% 
% \begin{frame}
% \frametitle{An Unfair Comparison}
% 
% \only<1>{
% \begin{figure}[b]
% \begin{center}
% \subfloat
% {
% \includegraphics[width=0.4\textwidth]
% {Figs/ControlSimulator/example_controlled_trajectories_id4.png}
% }
% \subfloat
% {
% \includegraphics[width=0.4\textwidth]
% {Figs/ControlSimulator/example_controlled_trajectories_id3.png}
% }
% \caption[]{Controlled trajectories using deterministic
% vs. stochastic control approaches. $\T$ is indicated with the red vertical
% line, $\tc, \b = [.75, 1.25]$.}
% \end{center}
% \end{figure}
% \note[item]{In A the performance of both
% control laws is essentially the same, but in B we see the advantage of the
% stochastic approach}
% }
% \only<2-3>{
% \begin{table}[b]
% \centering
% \begin{tabular}{lcc}
% Control Law & Empirical Error & Theoretical Error
% \\
% \hline
% Open-Loop Deterministic &  0.647  \onslide<3>{$\swarrow$} &  - \\
% Closed-Loop  Stochastic &  0.336  \onslide<3>{$\nwarrow$} &  0.285 \\
% \end{tabular}
% \caption{Realized performance of the different control laws. The Error is $\Exp
% \left[ (\T - \ts)^2 \right] $}
% \label{tab:realized_avg_errors_det_vs_stoch}
% \end{table}
% }
% \onslide<3>{
% \vskip 5pt
% \begin{center}
% Can we do better?
% \end{center}
% }
% \end{frame}




\section{Open-loop Stochastic Control - maximum principle}
\begin{frame}
\frametitle{Maximum Principle Solution}
Consider, $f(x,t)$, the forward transition density for $X_t$
\vskip 2pt
\only<1>{
\vskip 10pt
\begin{align*}
\Exp\left[
(\ts - \T \big)^2 \right]
=&
\int_0^\ts (t-\T)^2 \cdot \left( - \frac {\b^2} 2 \di_x f(\xth, t) \right) 
\intd{t}
\\
&+
\int_\xmin^\xth 
\Exp[\t^2|x, \amax] \cdot f(x,\T) \intd{x}
\end{align*}
\vskip 12pt
}
\only<2-3>{
\begin{center}
\begin{tikzpicture}[scale = 1.5]
\draw[->][dashed] (-.75, 1.) --
		node[at start, left]{$\xth$}
		node[at end, right] {$t$}
		(1.9,1.);
		
\draw[->][very thick] (.1, .5) --
		node[at start, left] 
			{$\Big[(t-\T)^2 \cdot \tfrac{-\b^2} 2 \di_x f\Big]$}
		node[at end , above]{``EARLY''}
		(.5,1.25);	

 \draw[->][dashed] 		
		(1.5,-.0) --
		node[at start, below]{$\T$}
		node[at end, above]{$x$}
		(1.5,1.25) ;
		
		\draw[->][very thick] (1.2, .0) --
		node[at start, left] {$\Big[\Exp[\t^2|x, \amax] \cdot f\Big]$}
		node[at end , right]{``LATE''}
		(1.8,.5);	 
\end{tikzpicture}
\end{center}
}
\vskip 1pt
\indent $f(x,t) \sim$ deterministic (it satisfies a Fokker-Planck equation)
\begin{align*}
\di_t f(x,t) =& -\di_x \left[(\a(t) - \frac{x}{\tc} )  f \right] + 
\frac{1}{2}\di^2_x \left[\b^2 f \right]
\end{align*}

\vskip 2pt
$\implies$ optimize using $f$
\vskip 2pt

\onslide<3>{
\begin{center}
But how do we optimize with PDEs?
\end{center}
}
\end{frame}

\begin{frame}
\frametitle{Lagrange Multipliers for PDEs}
\begin{align*}
J[\a]
=&
\int_\xmin^\xth 
\underbrace{\Exp[\t^2|x, \amax]}_{\Ttwo(x)} \cdot f(x,\T) \intd{x}
\\
&+ \int_0^\ts (t-\T)^2 \cdot \left( - \frac {\b^2} 2 \di_x f(\xth, t) \right) 
\intd{t}
\end{align*}
\\
\pause
\begin{align*}
&+ \int_0^\ts\int_\xmin^\xth
\underbrace{p(x,t)}_{\textrm{co-state}} \cdot \underbrace{(\di_t f -
(-\di_x \big[(\a(t) - \frac{x}{\tc} )  f \big] + \frac{1}{2}\di^2_x \left[\b^2 f \right] ))}_{\textrm{F-P eq'n}}
\intd{x}\intd{t}
\end{align*}
\pause
Optimality Principle:
$$
\frac{\delta J}{\delta \a} = 0 
$$
\end{frame}

\begin{frame}
\frametitle{Forward-Backward System}
\begin{minipage}{0.48\linewidth}
\begin{tikzpicture}[scale = .8]
\draw[->] (-.25,1) -- node[above]   {$f$} node[below]{State}
	(4.5,1) ;
\end{tikzpicture}
\includegraphics[width=0.75\textwidth]{Figs/FP_Adjoint/Stylized__f.png}
\end{minipage}
\vline 
\begin{minipage}{0.48\linewidth}
\begin{flushright}

\includegraphics[width=0.75\textwidth]{Figs/FP_Adjoint/Stylized__p.png}

\begin{tikzpicture}[scale =.8]
\draw[<-] (-.25,-1) -- node[below]   {$p$} node[above]{Co-State}
	(4.5,-1) ;
\end{tikzpicture}
\end{flushright}
\end{minipage}

% \hline
% \vskip 3pt
% \begin{equation*}
% \Big\{
%  \e  2 \cdot \a(t)
% + p f \Big|_\xmin
% - \int _\xmin^\xth p \cdot \di_x f \intd{x} 
% \Big\} = 0
% \quad \forall t \in [0,T]
% \end{equation*}
\end{frame}

\begin{frame}
\frametitle{Gradient Descent}
\begin{equation*}
\grad_\a [J] = 
 \e  2 \cdot \a(t)
+ p f \Big|_\xmin
- \int _\xmin^\xth p \cdot \di_x f \intd{x} 
\end{equation*}

\vskip 5pt

\begin{center}
We can use $\grad_\a J$ for gradient-based numerical optimization
\end{center}
\end{frame}

\begin{frame}
% \frametitle{A more fair comparison}
\frametitle{Simulation Comparison}
\only<1>{
\begin{figure}[h]
\begin{center}
\subfloat 
{
\label{fig:controlled_traj_ex1}
\includegraphics[width=0.40\textwidth]
{Figs/ControlSimulator/open_vs_closed_example_trajectories_id6.png}
}
\subfloat 
{
\includegraphics[width=0.40\textwidth]
{Figs/ControlSimulator/open_vs_closed_example_trajectories_id5.png}
}
\caption[]{
Controlled trajectories using open-loop vs.\
closed-loop control laws. $\T$ is indicated with the red 
vertical line, $\tc, \b = [.75, 1.25]$.} 
\end{center}
\end{figure}
}

\only<2> {\begin{table}[h] 
\centering
\begin{tabular}{lcc}
Control Law & Empirical Error & Theoretical Error \\
\hline
% Deterministic $(\b=0)$ &  0.621 & - \\
Open-Loop Stochastic & 0.356 & 0.382\\
Feedback Stochastic &  0.288& 0.285\\
\hline
\end{tabular}
\caption{Realized and theoretical performance of the different control laws. The empirical
performance is obtained using 128 sample paths.}
\label{tab:realized_avg_errors_det_vs_openloop_vs_stoch}
\end{table}
}

\end{frame}

\section*{Summary}

\begin{frame}
\frametitle<presentation>{Summary}
\begin{block}{Has been shown:}
  \begin{itemize}
    \item A stochastic control problem with minimal observations can be re-stated as a
deterministic optimal control problem using the transition density. 

	\item The deterministic optimal control problem can be solved using an
infinite-dimensional Pontryagin-type Principle and a simple gradient descent
procedure.
  \end{itemize}
\end{block}

\vskip0pt plus.5fill

\begin{block}{To be addressed:}
  \begin{itemize}
    \item Where to apply this?
  \end{itemize}
\end{block}

\end{frame}

\end{document}
