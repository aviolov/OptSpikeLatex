\documentclass{article}
\usepackage{amsmath}
\usepackage{amscd}
\usepackage{amssymb}
\usepackage{amsfonts}
\usepackage{amsthm}
\usepackage{amsfonts}
\usepackage{amsthm} 
%\usepackage{circuitikz}
% \usepackage{pgf}
% \usepackage{tikz} 
% \usetikzlibrary{arrows,snakes,backgrounds}
% \usetikz
% \usepackage{verbatim} 
\usepackage{subfig} 
\usepackage{graphicx}
\newtheorem*{rep}{Reply}
% \usepackage[super]{nth}
% \usepackage{appendix}
% \usepackage{listings}
% \usepackage{color}

% \usepackage{hyperref}
%\usepackage{url}

%\usepackage{cleveref}
% \usepackage{aviolov_style}
% \usepackage{local_style}
 
\begin{document}

\title{Stochastic Optimal Control of Single Neuron Spike Trains\\
--\\
Reply to Referees at the JNE} 

\author{Alexandre Iolov, 
Susanne Ditlevsen 
and
Andr\'e Longtin
}

\date{\today}

\maketitle 

%\maketitle

%\tableofcontents 

\vskip 20pt

We want to thank the 4 referees for the thoughtful and concrete points made on
our submission. We will attempt to answer all their concerns in a point-by-point
fashion. Below, we list the referees comments and then our replies immediately
following each comment.
\tableofcontents

\clearpage
\section{Referee: 1}
This is a nice study in which the authors study optimal control of spike timing in the presence of noise. I only have a few comments which the authors should consider:

\begin{enumerate}
\item the authors already compare their methodology with that of other in the
introduction, these statements should be moved to the discussion.
\begin{rep}
TODO
\end{rep}

\item can the authors quantify the performance of their algorithm (e.g. \% of
correct spikes) as a function of model parameters (e.g. noise intensity) in
different regimes. I believe that these will be more informative to the reader.
\begin{rep}
Note that this is a function of $\alpha_{max}$. So this really needs to be a two
dimensional plot with $\beta$ and $\alpha_{max}$ as independent axes.
See figure \ref{fig:target_error_as_func_of_beta} for how this might look,
where we investigate a Cartesian product of [Sub-Threshold, Supra-Threshold]
Target Trains being tracked by [Sub-Threshold, Supra-Threshold] tracking
trains, with showing three different error metrics as a function of $\beta$.
%\usepackage{graphics} is needed for \includegraphics
\begin{figure}[htp] 
\begin{center}
\subfloat[Sub-T Target / Sub-T Tracker]{
  \includegraphics[width=.99\textwidth]{Figs/TrainController/ControlError_Stats_cl_subthn_6_16_subt}
  }\\
  \subfloat[Sub-T Target / Supra-T Tracker]{
  \includegraphics[width=.99\textwidth]{Figs/TrainController/ControlError_Stats_cl_subthn_6_16_SUPT}
  }
  \\
\subfloat[Supra-T Target / Sub-T Tracker]{
  \includegraphics[width=.99\textwidth]{Figs/TrainController/ControlError_Stats_cl_supthn_8_16_subt}
  }\\
\subfloat[Supra-T Target / Supra-T Tracker]{
  \includegraphics[width=.99\textwidth]{Figs/TrainController/ControlError_Stats_cl_supthn_8_16_SUPT}
  }    
  \caption[labelInTOC]{Errors as a function of noise intensity for the
  CLOSED-LOOP controller. A 'correct' spike for the purposes of calculating
  'Percent Correct' is one that is within .1$\cdot \hat I$ of the target spike, where $\hat I$ is the mean ISI of
 the target train, here about .665}
  \label{fig:target_error_as_func_of_beta}
\end{center}
\end{figure} 
\end{rep}
\end{enumerate}

\clearpage
\section{Referee: 2}

In this manuscript, the problem of controlling the spike firing times in
response to stimulating waveforms was considered in the closed- and open-loop
control schemes. In particular, the authors have proposed a procedure for
determining the stimulating waveform in the open-loop situation which is useful
for clinical situations. The reviewer thinks that the results of the manuscript
are interesting and the manuscript would be well organized. However, some issues
need to be clarified before recommending the manuscript for publication in
Journal of Neural Engineering.


\subsection{Major issue}
Many readers would wonder if it would be appropriate to determine the
stimulating waveform on the basis of a criteria of the inter-spike intervals, as
the decision should be made from the knowledge on the all spike times,
i.e.,“spike trains”. The error of the spike firing times determined on the basis
of only ISIs seems to be accumulated as time goes. The authors have not cited
anything about how to determine the succeeding spike time interval for the
multi-spike control in the discussion section or anywhere else. The authors
should have clearly stated whether or not the spike time error could be canceled
by the succeeding spike time after the inaccurate preceding spike time given, in
the multi-spike control.
\begin{rep}
TODO
Put an explanation in beginning of section 5.2, and emphasize that quick
succeeding ISIs are the main bane. There should be no accumulation of errors
unless the controller is actually unable to catch up to an extremely high firing rate. 
But otherwise the controller is adaptive to the realized spike times such that
it is always targeting the original target train. 
\end{rep}

\subsection{Some other issues}
\begin{enumerate}
  \item 
--In l.6, Page 4 of 33, the authors stated “a non-negligible noise component
...”. What kind of noise would the authors assume? Is that a weak or strong one?
Is that Gaussian or not? Is that white or colored one?
\begin{rep}
By non-negligible noise component, we
mean a value of $\beta >> 0 $, while the noise is always Gaussian in the sense
that the SDE is driven by a Gaussian, white noise. 
\end{rep}


\item In 3.1 HJB equation subsection, the authors mentioned repeatedly the HJB 3
times in Eqs. (8), (10), and (13). It would be good enough to specify Eq.(13) so
as to simplify the problem under consideration.
\begin{rep}
TODO:
\end{rep}

\item In Page 11 of 33, Figure 2 shows w(x,t) and alpha(x,t) at t=0, t*/2, or t*.
However, the left column of (A, C, E, and G) looks redundant as Figure 3 shows
w(x,t) as functions of x and t. Many readers might want to look at the shape of
alpha(x,t) as functions of x and t, instead of the right column of (B, D, F, and
H) in Figure 2.
\begin{rep}
We can replace the surface plots for the value function with surface plots for
the control instead?
We make surface plots for $\alpha(x,t)$ IN ADDITION to fig. 3. An example is
shown in figure \ref{fig:HJB_control_surface}. Note that the aspect
of the new figure is rotated relative to the surface plots of $v(x,t)$, because
the different surfaces are best seen from different angles respectively.

TODO: Add to draft. 
%\usepackage{graphics} is needed for \includegraphics
\begin{figure}[htp]
\begin{center}
  \includegraphics[width=\textwidth]{Figs/HJB/Regimes_controlsurf.pdf}
  \caption[labelInTOC]{Surface plots for $\alpha(x,t)$ analogous to the surface
  plots for $v(x,t)$ in figure 3 in the draft}
  \label{fig:HJB_control_surface}
\end{center}
\end{figure} 
\end{rep}

\item In Simulations section, Page 18 of 33, the results of the naive control seem
unnecessary.
\begin{rep}
Both Susanne and Alex disagree with the ref. The point of the naive control as
far as Alex is concerned is to provide a base-case against which to compare the 'optimal'
stimulus. Our main justification is that if the optimal stimulus improves things
by 5,10\% then it is hard to justify its computational cost. 
\end{rep}


\item In the right column of Figure 6 in Page 22 of 33, 3 sample realizations of
x(t) and alpha(t) at epsilon=0.1 are depicted, but any explanation about those
graphs could not be found in section 6 or anywhere else, although the authors
have stated in the legend of Fig.6 “... see section 6)”.
\begin{rep}
TODO
\end{rep}

\item In l. 7 in Page 23 of 33, 'High Noise' should be deleted.
\begin{rep}
TODO: Here is the offending line
'In all cases,
the target train is obtained by a simulation of the Supra-Threshold High Noise
regime with no additional control, i.e.\ $\alpha(\cdot) = 0$.' It is actually on
page 22 in draft (perhaps there is an extra first page in the referee's copy).

I don't see why we should remove the High Noise qualifier. There is a
significant difference between a target train generated by 'Supra-Threshold High
Noise' vs.\ 'Supra-Threshold Low Noise'. 
\end{rep}

\item In the fourth paragraph in Page 23 of 33, the authors mentioned “In fig. 11,
we relax the bound constrains on the control $[\alpha_{min}, \alpha_{max}]$ from
[-2;2] to [-4;4].”. However they depicted the firing rates and dot raster plots only in
the case of the closed-loop control in Figure 11. They might want to show the
simulation results of the open-loop control scheme in relaxing the constraints
on the alpha's boundaries. Also, they might want to depict the sample
realizations of alpha(t) or the mean of alpha(t) as a function of time as well
as the firing rates and dot raster plots in the closed- and open-control
schemes.
\begin{rep}
TODO: Also do the relaxed bounds for open-loop???

Unfortunately, averaging $\alpha(X_t,t)$ over the different tracking trains
is not particularly sensible. In particular consider the case when one
tracking train has just spiked, while the other is delayed. Then the average
of the controls will be zero, even though one is max and the other is min. As
showing an average seems uninformative of what is going on. 

Instead we show one closed-loop / open-loop full-train realization of the
voltage / control signal, $X_t, \alpha(X_t, t)$ for a
single train in figure \ref{fig:tracking_train_trajectories}. In a sense this is
just 16 copies glued side-by-side of the realizations we see in section 5.1.

%\usepackage{graphics} is needed for \includegraphics
\begin{figure}[htp]
\begin{center}
\subfloat[Closed Loop]{
  \includegraphics[width=\textwidth]{Figs/TrainController/SingleTrain_Trajectories_SUPT_supthn_8_16_cl.pdf}
   }\\
  \subfloat[Open Loop]{ 
   \includegraphics[width=\textwidth]{Figs/TrainController/SingleTrain_Trajectories_SUPT_supthn_8_16_ol.pdf}
  }
  \caption[labelInTOC]{Full Tracking Train trajectories for $X_t$,
  $\alpha(X_t,t)$}
  \label{fig:tracking_train_trajectories}
\end{center}
\end{figure}

\end{rep}

\item In 3-4 lines before the section 6, “For example, in figs. 6 and 7” should be
replaced by “For example, in figs 7 and 8”.
\begin{rep}
TODO:
\end{rep}

\item In the last sentence of 5.2 Multi-Spike Control subsection in Page 23 of 33,
the authors stated “if the target interspike intervals are within reasonable
reach of the neuron, this is not a problem.” However, many readers would wonder
how much reasonable reach could be, since they might think of if the reach would
be longer or shorter than the refractory period.
\begin{rep}

Indeed, this is unprecise. Refractory period is clearly unreasonable, but more
than that, we consider any period that is shorter than the time it takes to
deterministically under maximum excitation. 

WE have added the following two sentences to the draft:
TODO:

\end{rep}
\item In Page 28 of 33. the left column of Figure 12 should be deleted because
of a duplication of the right column of Figure 2. Instead, the right column of Figure
12, alpha(x,t), should be depicted explicitly in 3 dim as functions of x and t.
\begin{rep}
TODO:
Are we going to make surface plots here again?
\end{rep}

\item In the left column of Figure 13 in Page 29 of 33, the traces were same as
those of Figure 4. The authors might want to refine how to show us the effect of
energy penalty.
\begin{rep}
The two traces are indeed the same! They are added here to compare against
the traces on the right, so that people don't flip back and forth for figures to
appreciate what has changed.
We will put both epsilon, open-loop curves on the same plots (red vs.
blue) see figure \ref{fig:epsilong_comparison_same_graph}.
%\usepackage{graphics} is needed for \includegraphics
\def \T {{ T}}
\begin{figure}[htp]
\begin{center}
  \includegraphics[width=\textwidth]{Figs/FP_Adjoint/Regimes_eps_comparison_joined.pdf}
  \caption[labelInTOC]{Effect of $\epsilon$ on the open-loop control.
  The deterministic time-trajectory of $\alpha(t)$ for each
  parameter regime as function of time, $t \in [0, \T]$.
  On the left, we use $\epsilon = 0.001$ and on the right $\epsilon =0.1$.
  The desired spike time is set to $\T=1.5$ and the bounds are $\alpha \in
  [-2,2]$. A)
   Supra-Threshold-Low-Noise
    B) 
   Supra-Threshold-High-Noise
    C)
   Sub-Threshold-Low-Noise 
    D)
   Sub-Threshold-High-Noise}
  \label{fig:epsilong_comparison_same_graph}
\end{center}
\end{figure}

\end{rep}

\item In the fourth paragraph of discussion section in Page 30 of 33, the
authors stated “the control scheme applies in the same manner in the Sub-Threshold
regime as well”. However, it would not be obvious for many readers as in the
sub-threshold regime, nobody knows if noise-enhanced stimuli elicit spikes or if
noise itself elicits spikes.
\begin{rep}
TODO: I don't understand the referee. Susanne pointed out that we need to
clarify that in all cases, the maximum value of the control puts the model in
the Supra-Threshold regime. This means that by using the control one can
always achieve Supra-Threshold behaviour. The Supra / Sub- distinction only
applies to the intrinsic (uncontrolled dynamics) and they are defined in the
absence of control.

TODO: Add ANdre's notes on Stochastic Resonance from his letter here. 
\end{rep}

\end{enumerate}

\clearpage
\section{Referee: 3}
The authors have considered two optimal control strategies on a single neuron
level using a noisy leaky integrate-and-fire model: Closed-loop control where
they use the Dynamic Programming technique, and Open-loop control where they use
a variation of Pontryagin's  Maximum Principle. They mention, as an advantage of
their work to others, that their control method is not limited by the magnitude
of the noise present in the model. They also consider the effect of energy
penalty and touch on the robustness of their algorithm. They have very clearly
demonstrated a background of recent literature and pointed out how the
mathematical formulation of their work is different.


This is a well written paper and, in my opinion, a nice extension to the
previous literature. I do however have some comments and questions that should
be addressed before the manuscript could be considered for publication in the
Journal of Neural Engineering (JNE).


\subsection{Major comments/questions:}
1. The most important shortcoming of this manuscript is the lack of adequate
discussion in applicability of the methods presented in a practical situation,
as would be most interesting for the majority of this journal's readers. In the
JNE, we are always looking for new ideas that are applicable in practice. The
authors need to discuss how these methods can be applied to a neuron, what are
the potential limitations (perhaps the choice of model), and how can one find a
way around those limitations. The authors allude to this in the discussion
section, but I think this needs a more elaborate discussion with possible
suggestions.
\begin{rep}
TODO:
This has to be intra-cellular injected current.

We require to have access to the output of the neuron.

We can use optogenetics\ldots
\end{rep}

2. Related to 1 above, the authors mention several highly sophisticated
network-level dependent interactions/diseases, such as brain-machine interface,
neural prostheses, the Parkinson's disease, and epilepsy, as potential
beneficiaries of their method. Given the fact the network-level activity in
neurons are much different than single neurons, and that the focus of the
current manuscript is on single neuron behaviour, I strongly suggest that the
authors revise the application sections of their paper and include other
physiological applications that are more relevant.
\begin{rep}
TODO: Include more single-cell applications.
\end{rep}
3. The authors need to acknowledge the difference between the Maximum Principle
as an optimal control theory that they have used and Dynamic Programming as an
efficient numerical method for finding the global minimum for an optimal control
problem. The way the manuscript is written, suggests that the authors use the
Maximum Principle only for the open-loop control. However, in the closed-loop
control problem, the fact that the control is bounded, means that one needs to
use the Maximum Principle to find the optimal control, just as the authors have
in (9). The Maximum Principle yields the control (9) and Dynamic Programming is
just a numerical method for solving for the optimal trajectories.
\begin{rep}
It is our impression that (Pontryagin's) Maximum Principle and (Bellman's)
Dynamic Programing approach are two distinct analytical methods for solving
Optimal Control problems.

The Maximum Principle is essentially a generalization of The Calculus of
Variations wherein it finds a critical point for the objective, akin to the
First Derivative Test in 1st year Calculus.
(refer to sources here)

Dynamic Programing is a sequential approach which builds up the optimal control
from the end-time recursively, using a value function to efficiently represent
the entire future payoff / cost associated with an incremental decision. (refer to
sources here)

There are connections between the two, most saliently, under certain conditions,
the adjoint function in the Maximum Principle is the spatial gradient of the
value function in Dynamic Programing (reference). But the two arise from
different paradigms and in turn give rise to different numerical methods. 

The difference in numerical methods is apparent in particular in that when we
apply Dynamic Programing, we can infer the optimal control right away, while for
the Maximum Principle, we are forced to use an iterative, gradient descent
procedure.


\end{rep}

4. How can these methods extend to networks of neurons efficiently? For
instance, Dynamic Programming suffers from the "curse of dimensionality". If one
were to make a coupled network of integrate-and-fire neurons, perhaps in an
effort to model network-level pathology, how can this network be solved
numerically in a time-efficient manner?
\begin{rep}
TODO:
Running N parallel neurons is not a problem, naturally.

However, controlling N coupled neurons is VERY challenging and certainly beyond
the limits of the techniques presented as is. 
\end{rep}

5. When epsilon is small, some of the controls that the authors have computed
are very similar to a bang-bang type control and even seem to have a net area
under curve of close to zero. I think it will help the comprehensiveness of the
paper if the authors made a somewhat elaborate connection between their
formalism to that of a bang-bang control formalism. And also explain why these
controls come out almost balanced in negative and positive values.
\begin{rep}
TODO:
1. Indeed as $\epsilon$ goes to zero, the optimization becomes bang-bang.

2. The occassional case where the controls are charge-balanced happens to be 
\end{rep}

\subsection{Minor comments/questions:}

Some of the control/mathematical jargon that is used in this paper could be
defined for the readers of JNE's better understanding. Examples of these include
"measurable with respect to the filtration of X", "transition density",
"artificial reflecting boundaries", "absorbing boundaries" 
\begin{rep}
TODO:
\end{rep}
The bounds on control should also appear in the minimization in (7)
\begin{rep}
TODO:
\end{rep}

P7,L44: Why do the authors assume for $x<<0$, w is not significantly affected?
What do they mean by $x<<0$? Looking at Figure 1, min(x) is usually around zero,
unless there is high noise, in which case it becomes in the order of -1 to -10
probably. Please elaborate a little.
\begin{rep}
TODO: Here is the offending paragraph redone:
 \def \xmin {{ x_{-}}}
  \def \amin {{ \alpha_{-}}}
  \def \tc {{ \tau_c}}
 'For $x \ll 0$, we assume that $w$ is not
significantly affected by the change in $x$, i.e.\ that $$ \partial_x
w(x_{-}, t) = 0 $$ for some lower boundary $x_{-}$. Such a boundary condition will be
justified if we impose artificial reflecting boundaries for the diffusion at
$\xmin$. So we need to determine for which $\xmin$ is a reflecting boundary
consistent with the dynamics. For example, we can take $\xmin$ to be two
standard deviations below the mean of the stationary distribution of the
maximally inhibited process, i.e.\ setting $\alpha=\amin$ in eq. [2].
 That is we set $\xmin = \tc(\mu + \amin) - 2 \beta /
\sqrt{\tc/2}$. We further enforce that $\xmin \leq -0.5$.'
\end{rep}

P10, Caption of Figure 2: "Note that the green and blue curves in F) are lying
on top of each other for all x." This is not true. Did the authors mean subplot
(D)? And can you explain why it is that the green and blue curves are on top of
each other?
\begin{rep}
 Indeed it is D) and thanks!

TODO: Add explanation for why this is so.
\end{rep}

The initial condition for the computations seems to be the point of previous
spike, x=0. Is that right? Please make note of this where ever appropriate for
clarity, for example, when showing Figure 4.
\begin{rep}
TODO: Add a reminder in Fig 4 that x=0 at time t=0.
\end{rep}

Can the authors elaborate on the noise that they implemented in their
simulations, and discuss the shape of the distributions in Figure 5, especially,
for the deterministic case?
\begin{rep}
TODO: Hmm\ldots We use standard Wiener-type driving force for the dynamics in
the stochastic. What is shown in Figure 5. is a function of the first passage
time of such a noise. 
\end{rep}
Figure 7: What does the solid horizontal line in the lower plot represent?
\begin{rep}
TODO: The dots above the horizontal line indicate the target spike times. The
dots below the horizontal line indicate the occurence of the tracking spike
times. 
\end{rep}

"i.e." usually has commas before and after. Also, at the beginning of sentence,
please consider using the words "That is" instead of "I.e." (P12,L19). For
consistency, please consider using either Supra- or Super-Threshold throughout.
\begin{rep}
Definitely yes on consistent use of 'Supra-' vs. 'Super-'. Thanks. 
TODO:  
\end{rep}





Typos and trivial mistakes: \\
P3,L06: setting $->$ settings\\
P4,L25: means $->$ is\\
P8,L22: optimisation $->$ optimization\\
P9,L14: be equal to $->$ equals\\
P10,L46: and on the optimal $->$ and the optimal\\
P15,L51: Sub-Threshold-Low-Noise regimes $- >$ Supra-Threshold-Low-Noise regimes
?
\begin{rep}
TODO:
\end{rep}

\clearpage
\section{Referee: 4}
In this manuscript, Iolov and colleagues develop and
test optimal algorithms for imposing predetermined spike patterns on simple
neural models.  The manuscript is generally clear and describes well the
advances over past methods.  However, I am left with the impression that this
first paper may be better suited for an applied math journal, with follow-up
work to be published in JNE.  Alternatively, the authors may wish to include
more applications-focused work in this manuscript.  This step would create a
paper that is better matched to JNE, at the possible cost of creating a
manuscript of unwieldy length.


\subsection{General points}
1. I am not convinced that this manuscript as presented is
a good match for JNE.  Certainly, stochastic optimal control is important for
neural devices, and the math is not beyond a significant fraction of JNE’s
readership.  The potential mismatch lies in the rhetoric, the assumptions, and
the examples given. 
\\1a. Rhetorically, the paper reads more like an applied
math paper than a neural engineering paper.  The authors are encouraged to place more
emphasis on the potential for application, especially in the results and
discussion sections. 
\begin{rep} TODO: \end{rep}

1b. For JNE, the authors should place more emphasis on robustness, which is
considered only very briefly in section 7, with no figures. In any application,
the experimenter is going to know very little about the underlying model, and
will have to build algorithms that work around this problem.

\begin{rep} TODO: 
We agree.
\end{rep}

1c. On a related front, the authors assume intracellular access  and
use highly oversimplified integrate-and-fire models.  It would be interesting to
test the effectiveness of the authors’ approach with more realistic models and
with explicitly extracellular forms of stimulation. 

\begin{rep} TODO:

Let's ignore the second part (on extracellular stimulation) and concentrate on
the first part (more complicated models).

TODO: Extracellular stimulation in the discussion only!
\end{rep}

2. The results and discussion sections could be improved by more context, to
connect specific findings to broader conclusions from the abstract and
introduction. 
\begin{rep} 
TODO: 
\end{rep}


\subsection{Specific points}
Abstract: The authors state that their goal was to design a control scheme to
achieve a target spike train with minimal damage to tissue.  I don’t think this
goal was achieved, because of the level of abstraction of the work.

\begin{rep} 
TODO:
 \end{rep}

p. 2, first par.  The goal mentioned by the authors of control via extracellular
stimulation does not seem to be achieved by the following work.  Also, the
examples noted by the authors (BMIs and prosthetics) are typically thought of as
problems for which one needs to interpret spike trains, not impose them.


\begin{rep} TODO: \end{rep} 

p. 2, par. 3.  The sentence beginning ``However the efficiency of the
numerics\ldots'' is nearly impossible to parse and understand.


\begin{rep} TODO: \end{rep}

p. 5, section 3.  The sentence beginning “Given a time t …” ends very awkwardly and should be rewritten.

\begin{rep} TODO: \end{rep}

p. 16.  I suggest putting Algorithm 1 in an appendix. 

\begin{rep} TODO: \end{rep}

Figure legends, especially for Figs. 7-10, begin awkwardly (e.g., “How good the
closed-loop controller is at hitting a target spike train.”)  I suggest
beginning the figure legends with declarative statements about the quality of
the outcome for open- and closed-loop algorithms in the low- and high-noise
cases.


\begin{rep} TODO: \end{rep}



 

% \bibliographystyle{plain}
% \bibliography{library}
\end{document}